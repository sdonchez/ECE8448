
%% SDonchezAnalyticalReview.tex
%% Stephen Donchez
%% Dr. Wang
%% ECE 8448-D L1
%% 6 May 2020

\documentclass[journal]{IEEEtran}

% enable citations
\usepackage{cite}

% enable graphics
\usepackage[pdftex]{graphicx}
%declare graphics location
\graphicspath{{./jpeg/}}
%declare extensions so they don't need to be explicitly defined
\DeclareGraphicsExtensions{.pdf,.jpeg,.png,.jpg}

%fix handling of long URLs in citations
\usepackage[hyphens]{url}
\usepackage[hidelinks]{hyperref}
\hypersetup{breaklinks=true}
\urlstyle{same}

\begin{document}

\title{ARM TrustZone in an SOC Environment}

\author{Stephen~Donchez,~\IEEEmembership{Member,~IEEE}%
\thanks{S. Donchez is with Villanova University, Villanova, PA 19085 USA 
\mbox{e-mail:sdonchez@villanova.edu}}%
\thanks{Manuscript received May 6, 2020}}%

% make the title area
\maketitle

\begin{abstract}
ARM's TrustZone platform is a well-known platform for the implementation of security
functionality in embedded systems. Although the platform is widely used in the larger
embedded development industry, it has been less widely studied with regards to its use in
Field Programmable Gate Array (FPGA) based embedded systems, which have widespread
popularity. This paper analyzes current research work being done in this field, which has 
demonstrated that there are numerous potential vulnerabilities exposed by such a system.
It then analyzes the countermeasures recommended by those conducting said research, as
well as proposes avenues for future work.
\end{abstract}

\begin{IEEEkeywords}
ARM TrustZone, Embedded System, System on a Chip, FPGA Security
\end{IEEEkeywords}

\section{Introduction}
\IEEEPARstart{I}{t} has rapidly become common knowledge within the technology industry, 
and to some extent in society at large, that the phrase “secure IoT device” is an oxymoron. 
Recent advances in computer technology have led to a massive surge of smart devices, with 
a particularly rapid growth in the consumer electronics sector. However, this rapid 
proliferation of such devices has led to an unfortunate discovery – many of them fail to 
adequately address security concerns, leading to vulnerabilities that are often harnessed 
by malicious actors.

As a result, the industry has begun to take a second look at security in embedded systems.
To this end, ARM ltd., the organization that oversees the development of the ARM processor
family, has introduced the ARM TrustZone platform, which seeks to offer ``an efficient, 
system-wide approach to security with hardware-enforced isolation built into the CPU.''
\cite{noauthor_trustzone_nodate} This platform effectively partitions the processor into
two discrete ``worlds'', one for secure operations and one for nonsecure (or normal) 
operations.

TrustZone has enjoyed tremendous success since its inception, and forms the basis for the
security of many common devices, including Android based smartphones. However, the concept
of Field Programmable Gate Array (FPGA) based System-on-a-Chip (SoC) devices introduces a
host of complexities into the implementation of a TrustZone enabled system. The
presence of in-situ reprogrammable hardware in such a system creates drastically increased
potential for malicious actors to attempt to compromise the integrity of said system.

Beyond the vulnerability created by having logic that can be altered present in such a
system, the FPGA development process introduces several additional concerns. First and
foremost, the industry as a rule relies heavily on third party logic designs, known as 
intellectual property (IP), for abstracting much of the intricacies of these complex 
systems. The presence of this IP brings with it a host of potential security concerns,
both as a result of maliciously compromised IP and also defects in otherwise genuine IP 
that may expose the larger system to exploitation. Furthermore, the nature of the FPGA
development process is heavily automated by a complex buildchain not dissimilar to a
compiler. This poses another avenue for attack - compromised buildchain software could
result in a device that performs as expected but presents additional avenues for
exploitation.

\subsection{Structure of the Paper}
This paper is structured into 6 major sections. The introduction in Section I seeks to
provide essential context as to the purpose of the ARM TrustZone platform, as well as the 
additional complexities introduced by using the platform in combination with a FPGA based 
SoC. Section II provides an explanation of the principles governing the operation of the 
TrustZone platform itself. Section III and IV discusses the state of current research on
the effective implementation of the ARM TrustZone into FPGA-based Embedded Systems, while 
sections V discusses potential avenues for further research. Section VI concludes 
the paper.

\subsection{A Comment on the State of the Industry}
Serious consideration of security in FPGA-based Embedded Systems is a novel field for
research. There are surprisingly few scholarly articles regarding this field, from an even
smaller pool of researchers. Accordingly, this paper will dar extensively from other works
on the TrustZone Architecture, many of which may not directly concern FPGA-based devices.

\section{Details of the TrustZone platform}
ARM's TrustZone technology exists to facilitate a Trusted Excecution Environment (TEE), in
order to enable the execution of "sensitive" tasks in such a way that maintains their
integrity without imposing massive constraints on system design. It facilitates this TEE
by means of partitioning the Processing System (PS) into two distinct "worlds" - the
secure world and the normal (interchangably referred to as non-secure) world. The
technology then polices access to secure world resources to ensure that non-secure tasks
aren't making erroneous (or malicious) accesses to secure world resources. This is
acomplished by the system's "Monitor Mode", which performs context switches between worlds
while also verifying access permissions and the validity of secure-world directed requests.

\subsection{Securing the AXI Bus}
TrustZone extends the concept of worlds beyond the PS and into peripherals,
via an extension to the AMBA3 Advanced eXtensible Interface (AXI) bus protocol. This
ensures that non-secure bus masters don't have access to secure slaves, while enabling
secure masters to access both secure and non-secure slave devices. Furthermore, this is
accomplished without requiring any modifications to the peripherals themselves, enabling
compatability with pre-existing devices.

The principle means of enforcing isolation on the AXI (system) bus is through a pair of
bits contained in the extended bus protocol. These bits are termed the Non-Secure (or NS)
bits, and one is dedicated to indicating the status of read operations, while the other is
reserved for write operations.\cite{benhani_security_2019} In the case of a non-secure bus
 master, these bits are permanently hardwired high, thereby preventing access to any 
 secure world resources. Meanwhile, secure world bus masters drive these lines using 
 internal logic, allowing them access to both secure and nonsecure resources.

Attempts to access secure world resources by a non-secure master result in a failure mode
that is implementation dependent. The sytem can either allow the request to fail silently
or can raise one of two errors: a slave error or a decode error. These errors are raised
on the BRESP (write) and RRESP (read) response signals, another crucial part of the 
TrustZone's AXI controls. \cite{benhani_security_2017}

\subsection{Securing the Advanced Peripheral Bus}
The AXI bus serves to connect the ARM processor with a variety of other components
internal to the system, much like the PCI family of busses commonly found in laptops and
desktops. However, most peripheral devices do not require the high rates of data transfer
(and the correspondingly stringent constraints) of the AXI specification. Instead, these
peripherals reside on the Advanced Peripheral Bus (APB), and are interfaced with the rest
of the system via the AXI-APB bridge.\cite{noauthor_arm_nodate}

As peripherals are often outside the scope of the designer's control, backwards
compatability is an impoerant factor in system design. Therefore, the APB specifications
do not allow for the implementation of an equivalent mechanism to the NS bit, as any such
implementation would not be compatible with prior versions of the AMBA protocol. To 
mitigate this, the TrustZone architecture allocates responsibility for security
enforcement of APB peripherals to the AXI-APB bridge itself.

The bridge features a set of one-bit input signals, one per peripheral, used to indicate
their respective security states. These signals can either be tied statically to limit
the peripheral to one world or the other, or can be driven by some other logic to allow
for dynamic reassignment. One example of such logic is ARM's TrustZone Protection 
Controller (TZPC) \cite{noauthor_arm_nodate}, although developers are also free to develop
their own control system.

\subsection{Securing the AXI Switch}
The nature of the AXI protocol is such that any non-trivial set of bus devices will
require the implementation of a routing mechanism to facilitate traffic flow. For Xilinx
based systems, this mechanism is the AXI Interconnect, a Xilinx provided piece of IP. 
\cite{noauthor_axi_2017} The interconnect provides a crossbar switch to enable routing 
between multiple masters, as well as a host of converters to handle different 
implementations of the protocol (data widths, clock speed, protocol versions, etc.). 
Altera offers a similar piece of IP, the Platform Designer (formerly Qsys) Interconnect,
which offers identical capabilities as well as an interface with their Avalon streaming
protocol. \cite{noauthor_intel_nodate}

The TrustZone documentation does not outline specific treatment of these interconnect
devices, however research explored in sections III and IV below demonstrate that
substantial vulnerabilities exist in this logic. As all bus traffic in the system must
flow through this IP, it is imperative that it be adequately secured.

\subsection{Securing the System Memory}
While all of the above aspects of the system concern the security of either peripheral
devices or other components likely located on the programmable logic (PL) side of the 
system, the main memory is another aspect of the system that requires careful 
consideration. Although the main memory is accessed via the AXI bus the same way any other
system component is reached, it is unique in that it contains both secure and non-secure
elements.

Like the AXI Interconnect, the implementation of the Memory Management Unit (MMU) is
manufacturer specific. However, both Intel/Altera and Xilinx's offerings provide
compatability with the TrustZone concept of NS bits, and verify security status as well as
the observance of traditional process isolation. \cite{gross_breaking_2019}

Additionally, the memory is heavily integrated into the caching system present in almost
all modern devices. As the fundamental premise of caching is speed of access, the tradeoff
between security and ease of access is highly present in this aspect of the system's
design. 

\section{Related Research in Embedded Systems}

\section{Related Research in FPGA-Based Embedded Systems}
In addition to the various vulnerabilities inherent in the TrustZone architecture outlined
above, the presence of a PL in a FPGA-based SoC introduces a number of
additional concerns. As mentioned earlier, critically small amounts of research
have been conducted in this field, however several papers have been presented that discuss
these vulnerabilities in depth. These papers focus primarily on specific vulnerabilities
present in SoC systems, such as those outlined below.

\subsection{Attacks on the AXI Bus NS Bits}
The concept of the NS bits (Formally the AWPROT and ARPROT signals) as they relate to the
AXI bus's overall security makes them an attractive target for compromise by actors 
seeking to violate the integrity of the Secure World. As Benhani et al. demonstrate in 
\cite{benhani_security_2019}, two potentia objectives can be achieved through malicious 
actions on these bits. By altering these bits such that they are no longer tied high 
(which otherwise prevents access to the Secure World), it is possible for a non-secure 
component to access Secure World Resources. Alternatively, by taking a previously variable
set of NS bits (such as those present on a Secure World component) and tying them high, it
is possible to implement a denial of service attack against that resource, impeding the 
ability of the Secure World routines that utilize that component to do so.

At first glance, a modification such as this seems infeasible, both due to the nature of
the access required to implement them as well as the high likelihood of discovery that a
suddenly nonfunctioning system component would imply. However, this is not necessarily the
case. Such vulnerabilities could be exploited easily by either a "rogue designer" or,
alternatively, by a compromised element of the buildchain. Furthermore, the nature of the
modification is not necessarily such that a device consistenyl behaves in a manner that is
indicative of the faulty configuration. Rather, the alteration of these signals could be
acomplished by more complex logic, such that the component being attacked behaves as
originally intended except under a precise set of circumstances.

This type of vulnerability has always been present in any kind of embedded system via a
compromised manufacturing process or inadequate supply chain security. Any of these
scenarios could allow for the introduction of a hardware trojan, such as is outlined in
\cite{bhunia_hardware_2014}. However, the presence of logic that can be reprogrammed, in
many cases from within the system itself, makes the potential for the manipulation of
these signals much greater.


\subsection{Attacks on the AXI Bus BRESP and RRESP Signals}
As an alternative to the modification of the NS bits, Benhani et al. also present an
attack vector that focuses on compromising the response signals BRESP and RRESP in 
\cite{benhani_security_2019}. As one might recall, these signals are used to indicate to
the requesting device (the Bus Master) if communication between the two devices is
permitted based on the NS Bit parameters and the nature of the slave device. Therefore, it
is conceivable that these response signals could be altered instead of the inputs to the
verification process. 

The risk of data leakage posed by the return of a false positive on these signals is
somewhat less prominent than in the case of modification of the NS bits, as it would
require the master to make an erroneous request that would normally result in an error. 
However, the risk of a denial of service attack remains present. Additionally, it is worth
mentioning that the poossibility of the malicious use of other logic to trigger such an
exploit under specific circumstances is equally as valid here as it is in the NS attacks.

\subsection{Attacks on the AXI Interconnect}
The AXI Interconnect (or Qsys interconnect for Intel/Altera FPGAs) is another prime target
for exploitation, as all bus traffic across the entire system must pass through the
crossbar as it flows from component to component. As all the signals (including the NS
bits and the response signals) must also flow through the crossbar, this represents
another potential means of modifying these bits to achieve the effects described above.

However, there is another, potentially more concerning avenue for compromise in the
interconnect logic. As opposed to a Denial of Service style attack, it is possible for a
malicious actor to implement a Man-in-the-Middle (MitM) attack by inserting a First In
First Out (FIFO) Queue into the crossbar, as is described in \cite{benhani_security_2017}
and \cite{benhani_security_2019}. Such data could then be conceivably accessed by a
malicious non-secure IP core, or could even be extracted using a side channel attack such
as those suggested in the literature \cite{bukasa_how_2018}.

\subsection{Direct Memory Access Attacks}
In \cite{gross_breaking_2019}, Gross et al. outline a number of potential attack vectors
which focus on compromising secure elements of the system's main memory by means of 
compromised hardware contained within the PL. As outlined in Section II,most of these
details are manufacturer specific, and \cite{gross_breaking_2019} focuses heavily on
Xilinx systems.

One particularly concerning vulnerability outlined in \cite{gross_breaking_2019} is the
potential to enable the execution of malicious binaries in the Secure World PS by 
compromising the mechanisms by which the Secure World verifies the integrity of its
executables. By manipulating PL components (As outlined in the preceding subsections),
the researchers were able to isolate and extract the routine that provided this
verification capability, and succeeded in modifying it such that it would always indicate
that a routine had been verified.They then were able to redeploy it to the system, and
proceeded to demonstrate the execution of their own untrusted binary in the Secure World.

The literature \cite{benhani_security_2019} outlines a similar memory vulnerability
utilizing the ARM Accelerator Coherency Port (ACP) in conjunction with the TrustZone
Components for configuring peripherals and memory. These devices, the TZPC outlined in 
section II.B and the TrustZone Address Space Controller (TZASC) designate access to
specific peripherals or 64 MB regions of memory, respectively. By manipulating the values
in the TZASC, a malicious actor can declare a previously secure memory region non-secure,
and then access the data contained therein from non-secure IP.

\section{Avenues for Future Research in TrustZone Enabled Systems}
%APB Exploitation
%Cache Exploitation
%Predefinition of IP permissions
%PL characterization and verification
%design time verification/signing


\section{Conclusion}
The conclusion goes here.

\section*{Acknowledgment}
The author would like to thank Dr. Xiaofang Wang, Villanova University, for her support in
the authoring of this paper.

\bibliographystyle{IEEEtran}

\bibliography{ZoteroExport}
\begin{IEEEbiography}[{
	\includegraphics[width=1in,height=1.25in,clip,keepaspectratio]{sdonchez}}]
	{Stephen Donchez}
(M'20) was born in Bethlehem, PA, USA in 1998. He anticipates receiving his B.S. in
computer engineering from Villanova University, Villanova, PA in 2020.

In the summer of 2018 he was a software engineering intern at Harris Coroporation 
(now L3Harris Technologies, Inc.). He returned to L3Harris for the summer of 2019 as a 
systems engineering intern, and anticpates returning to the same position for the summer
of 2020 at their facility in Clifton, NJ. His research interests include FPGAs, embedded
software development, and embedded systems with a focus on System-on-a-Chip technologies.

Mr. Donchez is a student member of the IEEE.
\end{IEEEbiography}

\end{document}


